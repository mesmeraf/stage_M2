\documentclass[12pt,a4paper]{report}

\usepackage[utf8]{inputenc}
\usepackage[francais]{babel}
\usepackage[T1]{fontenc}
\usepackage{amsmath}
\usepackage{amsfonts}
\usepackage{amssymb}
\RequirePackage[square,sort,comma,numbers]{natbib}
\RequirePackage[colorlinks=true,linkcolor=blue]{hyperref}
\usepackage[acronym]{glossaries}
\usepackage[usenames,dvipsnames]{xcolor}
\usepackage{setspace}

\author{Nicolas JEANNE}
\title{Rapport de stage de M2}
\date{11 mars 2015}

% formattage des entrées du glossaire
\renewcommand*{\glstextformat}[1]{\textcolor{Black}{#1}}
% création des éléments du glossaire
\newacronym{rep}{REP}{Repeated Extragenic Palindrome}
\newacronym{bime}{BIME}{Bacterial Interspesed Mosaic Element}
\makeglossaries

\begin{document}

\maketitle

\begin{doublespace}
\chapter*{Introduction}

En 1982, la découverte par Higgins et al. de nouveaux éléments génétiques communs dans les régions intercistroniques des opérons de Escherichia coli et Salmonella typhimurium a constitué le premier pas de la recherche sur les \gls{rep} \citep{Higgins1982}. En 1991, Gilson et al. ont mis en évidence l'organisation en clusters de ces REP \citep{Gilson1991}, ces clusters ont été caractérisés comme \gls{bime}. Chez E. coli en 1994, Bachelier et al. ont réussi à catégoriser les REP constituant les BIME en 2 types Y et Z, constituants 3 motifs Y, Z\textsuperscript{1}, Z\textsuperscript{2} \textcolor{red}{qu'en est il des Box C+ ?}\citep{Bachellier1994}.
 
Les REP consituent une part non négligeable du génome bactérien, chez E. coli K12 ou S. typhimurium elles représentent environ 1\% de celui-ci \citep{Gilson1991}. Nous les retrouvons chez de nombreux règnes bactériens, notamment chez les pathogènes humains tels que \textit{Escherichia coli, Salmonella enterica, Neisseria meningitidis, Mycobacterium tuberculosis et Pseudomonas aeruginosa} mais également chez des pathogènes des plantes comme \textit{Agrobacterium tumfaciens} ou chez des bactéries ubiquitaire, \textit{Deinococcus radiodurans} ou \textit{Pseudomonas putida} par exemple. \textcolor{red}{Le rôle exact des REP n'est pas clairement défini, des hypothèses sont avancées sur leur implication dans la régulation de l'expression des gènes}

\section*{Structure des REP}


\end{doublespace} 
% affichage du glossaire
\printglossary[type=\acronymtype ,title=Glossaire]

\bibliographystyle{unsrtnat}
\bibliography{biblio_rapport}



\end{document}