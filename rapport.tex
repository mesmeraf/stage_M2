\documentclass[12pt,a4paper]{report}

\usepackage[utf8]{inputenc}
\usepackage[francais]{babel}
\usepackage[T1]{fontenc}
\usepackage{amsmath}
\usepackage{amsfonts}
\usepackage{amssymb}
\RequirePackage[square,sort,comma,numbers]{natbib}
\RequirePackage{hyperref}
\usepackage[acronym]{glossaries}

\author{Nicolas JEANNE}

\title{Rapport de stage de M2}
\date{11 mars 2015}

\newacronym{rep}{REP}{Repeated Extragenic Palindrome}
\newacronym{bime}{BIME}{Bacterial Interspesed Mosaic Element}
\makeglossaries

\begin{document}

\maketitle




\section*{Introduction}

En 1982, la découverte par Higgins et al. de nouveaux éléments génétiques communs dans les régions intercistroniques des opérons de Escherichia coli et Salmonella typhimurium a constitué le premier pas de la recherche sur les \gls{rep} \citep{Higgins1982}. En 1991, Gilson et al. ont mis en évidence l'organisation possible en clusters de ces REP \citep{Gilson1991}, ces clusters ont été caractérisés comme \gls{bime}.



 
%Print the glossary
%\printglossaries
\printglossary[type=\acronymtype ,title=Abbreviations]

\bibliographystyle{plainnat}
\bibliography{biblio_rapport}



\end{document}