\documentclass[12pt,a4paper]{report}

\usepackage[utf8]{inputenc}
\usepackage[francais]{babel}
\usepackage[T1]{fontenc}
\usepackage{amsmath}
\usepackage{amsfonts}
\usepackage{amssymb}
\RequirePackage[square,sort,comma,numbers]{natbib}
\RequirePackage[colorlinks=true,linkcolor=blue]{hyperref}
\usepackage[acronym]{glossaries}
\usepackage[usenames,dvipsnames]{xcolor}
\usepackage{setspace}
\usepackage{graphicx}
\usepackage{float}
\usepackage[skip=2pt,font=scriptsize]{caption}


\author{Nicolas JEANNE}
\title{Rapport de stage de M2}
\date{11 mars 2015}

% formattage des entrées du glossaire
\renewcommand*{\glstextformat}[1]{\textcolor{Black}{#1}}
% création des éléments du glossaire
\newacronym{rep}{REP}{Repeated Extragenic Palindrome}
\newacronym{bime}{BIME}{Bacterial Interspesed Mosaic Element}
\makeglossaries

% Encadrement des figures
\floatstyle{boxed}
\restylefloat{figure}

\begin{document}

\maketitle

\begin{onehalfspace}
\chapter*{Introduction}
En 1982, la découverte par Higgins et al. de nouveaux éléments génétiques communs dans les régions intercistroniques des opérons de Escherichia coli et Salmonella typhimurium a constitué le premier pas de la recherche sur les \gls{rep} \citep{Higgins1982}. En 1991, Gilson et al. ont mis en évidence l'organisation en clusters de ces REP \citep{Gilson1991}, ces clusters ont été caractérisés comme \gls{bime}. Chez E. coli en 1994, Bachelier et al. ont réussi à catégoriser les REP constituant les BIME en 2 types Y et Z, constituants 3 motifs Y, Z\textsuperscript{1}, Z\textsuperscript{2}.
 
Les REP constituent une part non négligeable du génome bactérien, chez E. coli K12 ou S. typhimurium elles représentent environ 1\% de celui-ci \citep{Gilson1991}. Nous les retrouvons chez de nombreux règnes bactériens, notamment chez les pathogènes humains tels que \textit{Escherichia coli, Salmonella enterica, Neisseria meningitidis, Mycobacterium tuberculosis et Pseudomonas aeruginosa} mais également chez des pathogènes des plantes comme \textit{Agrobacterium tumefaciens} ou chez des bactéries ubiquitaire, \textit{Deinococcus radiodurans} ou \textit{Pseudomonas putida} par exemple. Les travaux précédents de l'équipe ont permis l'annotation des REP au sein du génome d'E. coli et de mettre en évidence le lien existant entre la prolifération des REP et le gène $tnpA_{REP}$ \citep{Weyder2013,Bosc2014}, ainsi que la reconstruction des états ancêtres des REP \citep{Bosc2014}.  \textcolor{red}{Le rôle exact des REP n'est pas clairement défini, des hypothèses sont avancées sur leur implication dans la régulation de l'expression des gènes, que ce soit en tant que terminateur ou comme site de reconnaissance des enzymes impliquées dans les mécanismes de la transcription.}

\section*{Structure des REP}
La taille des REP varie de 20 à 40 nucléotides, la classification Y, Z\textsuperscript{1}, Z\textsuperscript{2} est basée à la fois sur la taille de la séquence consensus de la REP ainsi que sur sa structure secondaire. Par convention, une REP en orientation inversée est nommée iREP (inversed REP) \citep{Ton-Hoang2012}. Un tétra-nucléotide caractéristique de séquence GTAC est présent à l'extrémité 5' des REP, sa séquence est CTAC en 3' pour les iREP. Les différentes classes de REP partagent des nucléotides conservés (Fig-1A)~\ref{fig:fig1}. La structure secondaire des REP est caractérisée par sa forme en tige-boucle, le caractère palindromique permet la formation de la tige malgré un mésappariement situé dans la partie centrale de la tige (Fig-1B)~\ref{fig:fig1}.

\textcolor{red}{parler de la composition des bime, nombre de répétitions, séquences S suivie de s/l/r pour BIME-2....}

\begin{figure}[h]
\centerline{\includegraphics[scale=2.0]{figures/fig1.jpg}}
\caption{\textbf{REP et BIME chez Escherichia coli. (A)} Séquences consensus Y, Z\textsuperscript{1} et Z\textsuperscript{2} des REP. Le tétra-nucléotide conservé GTAC est encadré en violet, le tétra-nucléotide conservé complémentaire CTAC est encadré en vert, les flèches rouges situent les zones d'appariement de la tige et les positions encadrées en noir sont les zones de mésappariement. Les positions conservées parmi les classes de REP sont en rouge, les positions variables en bleu. \textbf{(B)} Structure secondaire des REP. Les rectangles violets et verts représentent respectivement les tétra-nucléotides conservés GTAC pour les REP et CTAC pour les iREP. Les flèches noires indiquent l'orientation des REP. \textbf{(C)} Structures des BIME-1 et BIME-2. Les BIME sont composées des REP et de iREP séparées par un une séquence longue (L) ou courte (S), H-H et T-T dénotent respectivement une organisation tête à tête et queue à queue des REP. \citep{Ton-Hoang2012}.}\label{fig:fig1} 
\end{figure}


\chapter*{Matériel \& Méthodes}


\end{onehalfspace}


 
% affichage du glossaire
\printglossary[type=\acronymtype ,title=Glossaire]

\bibliographystyle{unsrtnat}
\bibliography{biblio_rapport}



\end{document}